% Options for packages loaded elsewhere
\PassOptionsToPackage{unicode}{hyperref}
\PassOptionsToPackage{hyphens}{url}
%
\documentclass[
]{article}
\usepackage{lmodern}
\usepackage{amsmath}
\usepackage{ifxetex,ifluatex}
\ifnum 0\ifxetex 1\fi\ifluatex 1\fi=0 % if pdftex
  \usepackage[T1]{fontenc}
  \usepackage[utf8]{inputenc}
  \usepackage{textcomp} % provide euro and other symbols
  \usepackage{amssymb}
\else % if luatex or xetex
  \usepackage{unicode-math}
  \defaultfontfeatures{Scale=MatchLowercase}
  \defaultfontfeatures[\rmfamily]{Ligatures=TeX,Scale=1}
\fi
% Use upquote if available, for straight quotes in verbatim environments
\IfFileExists{upquote.sty}{\usepackage{upquote}}{}
\IfFileExists{microtype.sty}{% use microtype if available
  \usepackage[]{microtype}
  \UseMicrotypeSet[protrusion]{basicmath} % disable protrusion for tt fonts
}{}
\makeatletter
\@ifundefined{KOMAClassName}{% if non-KOMA class
  \IfFileExists{parskip.sty}{%
    \usepackage{parskip}
  }{% else
    \setlength{\parindent}{0pt}
    \setlength{\parskip}{6pt plus 2pt minus 1pt}}
}{% if KOMA class
  \KOMAoptions{parskip=half}}
\makeatother
\usepackage{xcolor}
\IfFileExists{xurl.sty}{\usepackage{xurl}}{} % add URL line breaks if available
\IfFileExists{bookmark.sty}{\usepackage{bookmark}}{\usepackage{hyperref}}
\hypersetup{
  pdftitle={Detroit Schools},
  pdfauthor={Amelia Yurkofsky},
  hidelinks,
  pdfcreator={LaTeX via pandoc}}
\urlstyle{same} % disable monospaced font for URLs
\usepackage[margin=1in]{geometry}
\usepackage{color}
\usepackage{fancyvrb}
\newcommand{\VerbBar}{|}
\newcommand{\VERB}{\Verb[commandchars=\\\{\}]}
\DefineVerbatimEnvironment{Highlighting}{Verbatim}{commandchars=\\\{\}}
% Add ',fontsize=\small' for more characters per line
\usepackage{framed}
\definecolor{shadecolor}{RGB}{248,248,248}
\newenvironment{Shaded}{\begin{snugshade}}{\end{snugshade}}
\newcommand{\AlertTok}[1]{\textcolor[rgb]{0.94,0.16,0.16}{#1}}
\newcommand{\AnnotationTok}[1]{\textcolor[rgb]{0.56,0.35,0.01}{\textbf{\textit{#1}}}}
\newcommand{\AttributeTok}[1]{\textcolor[rgb]{0.77,0.63,0.00}{#1}}
\newcommand{\BaseNTok}[1]{\textcolor[rgb]{0.00,0.00,0.81}{#1}}
\newcommand{\BuiltInTok}[1]{#1}
\newcommand{\CharTok}[1]{\textcolor[rgb]{0.31,0.60,0.02}{#1}}
\newcommand{\CommentTok}[1]{\textcolor[rgb]{0.56,0.35,0.01}{\textit{#1}}}
\newcommand{\CommentVarTok}[1]{\textcolor[rgb]{0.56,0.35,0.01}{\textbf{\textit{#1}}}}
\newcommand{\ConstantTok}[1]{\textcolor[rgb]{0.00,0.00,0.00}{#1}}
\newcommand{\ControlFlowTok}[1]{\textcolor[rgb]{0.13,0.29,0.53}{\textbf{#1}}}
\newcommand{\DataTypeTok}[1]{\textcolor[rgb]{0.13,0.29,0.53}{#1}}
\newcommand{\DecValTok}[1]{\textcolor[rgb]{0.00,0.00,0.81}{#1}}
\newcommand{\DocumentationTok}[1]{\textcolor[rgb]{0.56,0.35,0.01}{\textbf{\textit{#1}}}}
\newcommand{\ErrorTok}[1]{\textcolor[rgb]{0.64,0.00,0.00}{\textbf{#1}}}
\newcommand{\ExtensionTok}[1]{#1}
\newcommand{\FloatTok}[1]{\textcolor[rgb]{0.00,0.00,0.81}{#1}}
\newcommand{\FunctionTok}[1]{\textcolor[rgb]{0.00,0.00,0.00}{#1}}
\newcommand{\ImportTok}[1]{#1}
\newcommand{\InformationTok}[1]{\textcolor[rgb]{0.56,0.35,0.01}{\textbf{\textit{#1}}}}
\newcommand{\KeywordTok}[1]{\textcolor[rgb]{0.13,0.29,0.53}{\textbf{#1}}}
\newcommand{\NormalTok}[1]{#1}
\newcommand{\OperatorTok}[1]{\textcolor[rgb]{0.81,0.36,0.00}{\textbf{#1}}}
\newcommand{\OtherTok}[1]{\textcolor[rgb]{0.56,0.35,0.01}{#1}}
\newcommand{\PreprocessorTok}[1]{\textcolor[rgb]{0.56,0.35,0.01}{\textit{#1}}}
\newcommand{\RegionMarkerTok}[1]{#1}
\newcommand{\SpecialCharTok}[1]{\textcolor[rgb]{0.00,0.00,0.00}{#1}}
\newcommand{\SpecialStringTok}[1]{\textcolor[rgb]{0.31,0.60,0.02}{#1}}
\newcommand{\StringTok}[1]{\textcolor[rgb]{0.31,0.60,0.02}{#1}}
\newcommand{\VariableTok}[1]{\textcolor[rgb]{0.00,0.00,0.00}{#1}}
\newcommand{\VerbatimStringTok}[1]{\textcolor[rgb]{0.31,0.60,0.02}{#1}}
\newcommand{\WarningTok}[1]{\textcolor[rgb]{0.56,0.35,0.01}{\textbf{\textit{#1}}}}
\usepackage{longtable,booktabs}
\usepackage{calc} % for calculating minipage widths
% Correct order of tables after \paragraph or \subparagraph
\usepackage{etoolbox}
\makeatletter
\patchcmd\longtable{\par}{\if@noskipsec\mbox{}\fi\par}{}{}
\makeatother
% Allow footnotes in longtable head/foot
\IfFileExists{footnotehyper.sty}{\usepackage{footnotehyper}}{\usepackage{footnote}}
\makesavenoteenv{longtable}
\usepackage{graphicx}
\makeatletter
\def\maxwidth{\ifdim\Gin@nat@width>\linewidth\linewidth\else\Gin@nat@width\fi}
\def\maxheight{\ifdim\Gin@nat@height>\textheight\textheight\else\Gin@nat@height\fi}
\makeatother
% Scale images if necessary, so that they will not overflow the page
% margins by default, and it is still possible to overwrite the defaults
% using explicit options in \includegraphics[width, height, ...]{}
\setkeys{Gin}{width=\maxwidth,height=\maxheight,keepaspectratio}
% Set default figure placement to htbp
\makeatletter
\def\fps@figure{htbp}
\makeatother
\setlength{\emergencystretch}{3em} % prevent overfull lines
\providecommand{\tightlist}{%
  \setlength{\itemsep}{0pt}\setlength{\parskip}{0pt}}
\setcounter{secnumdepth}{-\maxdimen} % remove section numbering
\ifluatex
  \usepackage{selnolig}  % disable illegal ligatures
\fi

\title{Detroit Schools}
\author{Amelia Yurkofsky}
\date{5/6/2021}

\begin{document}
\maketitle

\#\# Introduction

The objective of this analysis is to determine characteristics of
Detroit schools that had elevated results during 2016 lead water
testing. I'll use four publicly available data sets: The first three are
publicly available data from the City of Detroit's `Detroit's Open Data
Portal' (two of these - `Charter and EEA Schools Water Testing Results'
and `DPS Water Testing Results' - provide information on results of
tests for lead in drinking water at 209 Detroit Public Schools that were
conducted in 2016, the third, `Data from State of Michigan's Center for
Educational Performance and Information,' provides information on the
location and characteristics of these schools, and the final is a
publicly available data set with information on neighborhood poverty and
income levels. (see Appendix for links to all sources).

\begin{Shaded}
\begin{Highlighting}[]
\FunctionTok{suppressPackageStartupMessages}\NormalTok{(}\FunctionTok{library}\NormalTok{(readr))}
\FunctionTok{suppressPackageStartupMessages}\NormalTok{(}\FunctionTok{library}\NormalTok{(dplyr))}

\CommentTok{\#Import water testing results (downloaded from Detroit Open Data Portal 5/8/2021)}
\FunctionTok{library}\NormalTok{(readr)}
\NormalTok{Water\_Testing\_Results\_charter }\OtherTok{\textless{}{-}} \FunctionTok{read\_csv}\NormalTok{(}\StringTok{"Charter\_and\_EEA\_Schools\_Water\_Testing\_Results.csv"}\NormalTok{)}
\NormalTok{Water\_Testing\_Results\_charter}\SpecialCharTok{$}\NormalTok{type }\OtherTok{\textless{}{-}} \StringTok{"Charter"}
\NormalTok{Water\_Testing\_Results\_dps }\OtherTok{\textless{}{-}} \FunctionTok{read\_csv}\NormalTok{(}\StringTok{"DPS\_Water\_Testing\_Results.csv"}\NormalTok{)}
\NormalTok{Water\_Testing\_Results\_dps}\SpecialCharTok{$}\NormalTok{type }\OtherTok{\textless{}{-}} \StringTok{"DPS"}
\NormalTok{water\_testing }\OtherTok{\textless{}{-}} \FunctionTok{rbind}\NormalTok{(Water\_Testing\_Results\_charter[,}\FunctionTok{c}\NormalTok{(}\DecValTok{1}\NormalTok{,}\DecValTok{2}\NormalTok{,}\DecValTok{7}\NormalTok{)], Water\_Testing\_Results\_dps[,}\FunctionTok{c}\NormalTok{(}\DecValTok{1}\NormalTok{,}\DecValTok{2}\NormalTok{,}\DecValTok{7}\NormalTok{)])}

\CommentTok{\#Import school information (downloaded from Detroit Open Data Portal 5/8/2021)}
\NormalTok{schools\_info }\OtherTok{\textless{}{-}} \FunctionTok{read\_csv}\NormalTok{(}\StringTok{"Schools\_\_All\_Schools\_\_2018\_{-}\_2019\_.csv"}\NormalTok{)}
\end{Highlighting}
\end{Shaded}

\#\# Preprocessing

In pre-processing I'll do the following: 1) Dropping unneeded variables,
2) Merging and matching/ fuzzy-matching the data sets by school name, 3)
Feature specification, 4) Tackling the class imbalance of the outcome
variable.

After dropping redundant and unneeded variables, I'll focus on the
following predictors: Years the school is open, type of school (public
or charter), and number of grades. Later I'll merge in the percent of
the population under 100\% of the federal poverty ratio by zip code.

\begin{Shaded}
\begin{Highlighting}[]
\CommentTok{\#list of variables to keep}
\NormalTok{vars\_keep }\OtherTok{\textless{}{-}} \FunctionTok{names}\NormalTok{(schools\_info) }\SpecialCharTok{\%in\%} \FunctionTok{c}\NormalTok{(}\StringTok{"EntityOfficialName"}\NormalTok{, }\StringTok{"EntityFIPSCode"}\NormalTok{, }\StringTok{"EntityOpenDate"}\NormalTok{, }\StringTok{"EntityPhysicalZip4"}\NormalTok{, }\StringTok{"type"}\NormalTok{,}
                                        \StringTok{"EntityActualGrades"}\NormalTok{, }\StringTok{"EntityGeographicLEADistrictOffi"}\NormalTok{,}\StringTok{"EntityAuthorizedEducationalSett"}\NormalTok{)}

\CommentTok{\#drop unneeded variables}
\NormalTok{schools\_info }\OtherTok{\textless{}{-}}\NormalTok{ schools\_info[vars\_keep]}
\end{Highlighting}
\end{Shaded}

The first attempt at merging the school water testing results and school
information by school name yielded 87 (out of 209) matches.

\begin{Shaded}
\begin{Highlighting}[]
\CommentTok{\#merge}
\NormalTok{matched }\OtherTok{\textless{}{-}} \FunctionTok{inner\_join}\NormalTok{(water\_testing, schools\_info, }\AttributeTok{by =} \FunctionTok{c}\NormalTok{(}\StringTok{"school\_name"}\OtherTok{=}\StringTok{"EntityOfficialName"}\NormalTok{)) }\CommentTok{\#matched}
\NormalTok{unmatch }\OtherTok{\textless{}{-}} \FunctionTok{anti\_join}\NormalTok{(water\_testing, schools\_info, }\AttributeTok{by =} \FunctionTok{c}\NormalTok{(}\StringTok{"school\_name"}\OtherTok{=}\StringTok{"EntityOfficialName"}\NormalTok{))[,}\DecValTok{1}\NormalTok{] }
\NormalTok{unmatch2 }\OtherTok{\textless{}{-}} \FunctionTok{anti\_join}\NormalTok{(schools\_info, water\_testing, }\AttributeTok{by =} \FunctionTok{c}\NormalTok{(}\StringTok{"EntityOfficialName"}\OtherTok{=}\StringTok{"school\_name"}\NormalTok{))[,}\DecValTok{1}\NormalTok{]}
\NormalTok{unmatch2 }\OtherTok{\textless{}{-}} \FunctionTok{subset}\NormalTok{(unmatch2, }\SpecialCharTok{!}\FunctionTok{is.na}\NormalTok{(unmatch2}\SpecialCharTok{$}\NormalTok{EntityOfficialName))}
\NormalTok{unmatch2 }\OtherTok{\textless{}{-}}\NormalTok{ unmatch2 }\SpecialCharTok{\%\textgreater{}\%} \FunctionTok{rename}\NormalTok{(}\AttributeTok{school\_name =}\NormalTok{ EntityOfficialName)}
\FunctionTok{dim}\NormalTok{(matched)[}\DecValTok{1}\NormalTok{]}
\end{Highlighting}
\end{Shaded}

\begin{verbatim}
## [1] 87
\end{verbatim}

Because there are a significant number of school names that don't match,
I'll attempt a process for `fuzzy matching' where I'll look for the
number of words in common between the two lists of school names. Fist
I'll turn each vector of school names into bags-of-words using the qdap
package. This results in two lists of list, with each sublist containing
the words of the school name without punctuation.

\begin{Shaded}
\begin{Highlighting}[]
\FunctionTok{suppressPackageStartupMessages}\NormalTok{(}\FunctionTok{library}\NormalTok{(qdap))}

\CommentTok{\#using qdap\textquotesingle{}s "bag\_o\_words" function to turn each school name into bag of words}
\NormalTok{unmatch\_bag }\OtherTok{\textless{}{-}}\FunctionTok{sapply}\NormalTok{(unmatch}\SpecialCharTok{$}\NormalTok{school\_name, bag\_o\_words, }\AttributeTok{simplify =} \ConstantTok{TRUE}\NormalTok{)}
\NormalTok{unmatch2\_bag }\OtherTok{\textless{}{-}} \FunctionTok{sapply}\NormalTok{(unmatch2}\SpecialCharTok{$}\NormalTok{school\_name, bag\_o\_words, }\AttributeTok{simplify =} \ConstantTok{TRUE}\NormalTok{)}
\CommentTok{\#look at first 5 school names and their associated bag{-}of{-}words}
\FunctionTok{head}\NormalTok{(unmatch\_bag) }
\end{Highlighting}
\end{Shaded}

\begin{verbatim}
## $`Al-Furqan-Quran Academy`
## [1] "al"      "furqan"  "quran"   "academy"
## 
## $`Allen Academy`
## [1] "allen"   "academy"
## 
## $`ASA Higher Learning Preparatory School & PreK`
## [1] "asa"         "higher"      "learning"    "preparatory" "school"     
## [6] "prek"       
## 
## $`Cesar Chavez Academy - Martin`
## [1] "cesar"   "chavez"  "academy" "martin" 
## 
## $`Cesar Chavez Academy - Waterman`
## [1] "cesar"    "chavez"   "academy"  "waterman"
## 
## $`Cesar Chavez Academy East Elementary School - Maxwell`
## [1] "cesar"      "chavez"     "academy"    "east"       "elementary"
## [6] "school"     "maxwell"
\end{verbatim}

Now I'll find the length of the intersection each possible pair of
bag-of-words. This will provide the number of words that each bag has in
common. Then I'll convert the results to a data frame.

\begin{Shaded}
\begin{Highlighting}[]
\CommentTok{\#count the number of shared words between each possible school name pairs }
\NormalTok{data }\OtherTok{\textless{}{-}} \FunctionTok{lapply}\NormalTok{(unmatch\_bag, }\ControlFlowTok{function}\NormalTok{(x) }\FunctionTok{lapply}\NormalTok{(unmatch2\_bag, }\ControlFlowTok{function}\NormalTok{(y) }\FunctionTok{length}\NormalTok{(}\FunctionTok{intersect}\NormalTok{(x,y))))}

\CommentTok{\#taking a look at the first five intersections for Cesar Chaves Academy {-} Waterman}
\CommentTok{\#we can see no words were in common for each of the first five school names}
\FunctionTok{head}\NormalTok{(data[[}\StringTok{"Cesar Chavez Academy {-} Waterman"}\NormalTok{]]) }
\end{Highlighting}
\end{Shaded}

\begin{verbatim}
## $`Primary Colors Early Childhood Center LLC`
## [1] 0
## 
## $`Kiddie University Child Care Center`
## [1] 0
## 
## $`Towering Glory Ascension Ministry Church`
## [1] 0
## 
## $`Exceptional Deliverance Ministries`
## [1] 0
## 
## $`College for Creative Studies`
## [1] 0
## 
## $`CAREER CHILD CARE CTR`
## [1] 0
\end{verbatim}

\begin{Shaded}
\begin{Highlighting}[]
\CommentTok{\#create data frame of results}
\NormalTok{data2 }\OtherTok{\textless{}{-}} \FunctionTok{data.frame}\NormalTok{(}\FunctionTok{do.call}\NormalTok{(rbind, data))}
\FunctionTok{names}\NormalTok{(data2) }\OtherTok{\textless{}{-}}\NormalTok{ unmatch2}\SpecialCharTok{$}\NormalTok{school\_name}
\end{Highlighting}
\end{Shaded}

Now I'll find the pair of school names with the highest number of words
in common. Because there are likely some schools that won't match I'll
rule these out by ensuring I only match schools with one unique highest
number of matches. To do this I'll take three highest numbers -- the
first, last, and a random. If these return the same school name, I'll
consider this the best and final match.

\begin{Shaded}
\begin{Highlighting}[]
\CommentTok{\#finding schools names with best matches}
\NormalTok{unmatch}\SpecialCharTok{$}\NormalTok{matches }\OtherTok{\textless{}{-}} \FunctionTok{colnames}\NormalTok{(data2)[}\FunctionTok{max.col}\NormalTok{(data2,}\AttributeTok{ties.method=}\StringTok{"first"}\NormalTok{)]}
\NormalTok{unmatch}\SpecialCharTok{$}\NormalTok{matches2 }\OtherTok{\textless{}{-}} \FunctionTok{colnames}\NormalTok{(data2)[}\FunctionTok{max.col}\NormalTok{(data2,}\AttributeTok{ties.method=}\StringTok{"last"}\NormalTok{)]}
\NormalTok{unmatch}\SpecialCharTok{$}\NormalTok{matches3 }\OtherTok{\textless{}{-}} \FunctionTok{colnames}\NormalTok{(data2)[}\FunctionTok{max.col}\NormalTok{(data2,}\AttributeTok{ties.method=}\StringTok{"random"}\NormalTok{)]}
\NormalTok{unmatch}\SpecialCharTok{$}\NormalTok{final\_match }\OtherTok{\textless{}{-}} \FunctionTok{ifelse}\NormalTok{((unmatch}\SpecialCharTok{$}\NormalTok{matches}\SpecialCharTok{!=}\NormalTok{ unmatch}\SpecialCharTok{$}\NormalTok{matches2) }\SpecialCharTok{\&}
\NormalTok{                              (unmatch}\SpecialCharTok{$}\NormalTok{matches2}\SpecialCharTok{!=}\NormalTok{unmatch}\SpecialCharTok{$}\NormalTok{matches3) }\SpecialCharTok{\&}
\NormalTok{                                (unmatch}\SpecialCharTok{$}\NormalTok{matches3}\SpecialCharTok{!=}\NormalTok{ unmatch}\SpecialCharTok{$}\NormalTok{matches), }\ConstantTok{NA}\NormalTok{, unmatch}\SpecialCharTok{$}\NormalTok{matches)}
\CommentTok{\#keep only the best match}
\NormalTok{unmatch }\OtherTok{\textless{}{-}}\NormalTok{ unmatch[,}\FunctionTok{c}\NormalTok{(}\DecValTok{1}\NormalTok{,}\DecValTok{5}\NormalTok{)]}
\NormalTok{unmatch }\OtherTok{\textless{}{-}} \FunctionTok{subset}\NormalTok{(unmatch, }\SpecialCharTok{!}\FunctionTok{is.na}\NormalTok{(unmatch}\SpecialCharTok{$}\NormalTok{final\_match))}
\end{Highlighting}
\end{Shaded}

This `fuzzy matching' technique yielded us another 96 matches, resulting
in a total of 176 of the 209 water testing result schools sites matched
to school location information. Spot-checking the data shows that while
not every fuzzy-match is plausible, the vast majority are likely
accurate.

\begin{Shaded}
\begin{Highlighting}[]
\CommentTok{\#How many new matches did we get?}
\FunctionTok{dim}\NormalTok{(unmatch)[}\DecValTok{1}\NormalTok{]}
\end{Highlighting}
\end{Shaded}

\begin{verbatim}
## [1] 96
\end{verbatim}

\begin{Shaded}
\begin{Highlighting}[]
\CommentTok{\#Spot{-}checking the matches}
\FunctionTok{sample\_n}\NormalTok{(unmatch, }\DecValTok{10}\NormalTok{)}
\end{Highlighting}
\end{Shaded}

\begin{verbatim}
## # A tibble: 10 x 2
##    school_name                           final_match                            
##    <chr>                                 <chr>                                  
##  1 Twain School for Scholars             "Henry Ford Academy: School for Creati~
##  2 ASA Higher Learning Preparatory Scho~ "ASA Higher Learning Preparatory"      
##  3 GEE White Academy & PreK              "GEE White Academy"                    
##  4 Davis Preparatory Academy & Preschool "Davis Preparatory Academy"            
##  5 Nataki Talibah Schoolhouse of Detroit "Detroit Academy of Arts and Sciences ~
##  6 Dove Academy of Detroit & Child Care  "Dove Academy of Detroit"              
##  7 U Prep Academy Mark Murray Campus     "Capstone Academy Charter School (SDA)~
##  8 Ellington Conservatory of Music & Ar~ "Edward \\\"Duke\\\" Ellington at Beck~
##  9 Old Redford Academy - High School     "Old Redford Academy - High"           
## 10 Detroit Christo Rey High School       "Detroit Cristo Rey High School"
\end{verbatim}

\begin{Shaded}
\begin{Highlighting}[]
\CommentTok{\#joining with existing matches}
\NormalTok{matched2 }\OtherTok{\textless{}{-}} \FunctionTok{inner\_join}\NormalTok{(unmatch, water\_testing, }\AttributeTok{by =} \StringTok{"school\_name"}\NormalTok{) }\CommentTok{\#matched}
\NormalTok{matched2 }\OtherTok{\textless{}{-}} \FunctionTok{inner\_join}\NormalTok{(matched2, schools\_info, }\AttributeTok{by =} \FunctionTok{c}\NormalTok{(}\StringTok{"final\_match"} \OtherTok{=} \StringTok{"EntityOfficialName"}\NormalTok{))}
\NormalTok{matched2}\SpecialCharTok{$}\NormalTok{final\_match }\OtherTok{\textless{}{-}} \ConstantTok{NULL}
\NormalTok{matched }\OtherTok{\textless{}{-}} \FunctionTok{rbind}\NormalTok{(matched, matched2)}
\end{Highlighting}
\end{Shaded}

Now I'll merge in `Income to Poverty Ratios in Michigan by Zip Code
Tabulation Area, 2013' available from Data Driven Detroit (see Appendix
for link). The data gives us the percentage of the Michigan population,
by zip code, under 100\% of the federal poverty ratio.

\begin{Shaded}
\begin{Highlighting}[]
\NormalTok{poverty\_ratios\_2013 }\OtherTok{\textless{}{-}} \FunctionTok{read\_csv}\NormalTok{(}\StringTok{"\textasciitilde{}/911\_calls/Income\_to\_Poverty\_Ratios\_in\_Michigan\_by\_Zip\_Code\_Tabulation\_Area,\_2013.csv"}\NormalTok{)}
\NormalTok{povratio }\OtherTok{\textless{}{-}}\NormalTok{ poverty\_ratios\_2013[,}\FunctionTok{c}\NormalTok{(}\DecValTok{3}\NormalTok{,}\DecValTok{14}\NormalTok{)] }

\FunctionTok{hist}\NormalTok{(povratio}\SpecialCharTok{$}\NormalTok{Pct\_U100)  }
\end{Highlighting}
\end{Shaded}

\includegraphics{detroit-schools_files/figure-latex/unnamed-chunk-8-1.pdf}

\begin{Shaded}
\begin{Highlighting}[]
\CommentTok{\#cleaning zip codes}
\NormalTok{matched}\SpecialCharTok{$}\NormalTok{zip\_code }\OtherTok{\textless{}{-}} \FunctionTok{substr}\NormalTok{(matched}\SpecialCharTok{$}\NormalTok{EntityPhysicalZip4, }\DecValTok{1}\NormalTok{, }\DecValTok{5}\NormalTok{)}
\NormalTok{matched}\SpecialCharTok{$}\NormalTok{zip\_code }\OtherTok{\textless{}{-}} \FunctionTok{as.numeric}\NormalTok{(matched}\SpecialCharTok{$}\NormalTok{zip\_code)}
\CommentTok{\#merge}
\NormalTok{total }\OtherTok{\textless{}{-}} \FunctionTok{inner\_join}\NormalTok{(matched, povratio, }\AttributeTok{by =} \FunctionTok{c}\NormalTok{(}\StringTok{"zip\_code"}\OtherTok{=}\StringTok{"ZCTA5CE10"}\NormalTok{))}
\end{Highlighting}
\end{Shaded}

\# Exploratory Data Analysis

The histogram below shows the distribution of the population by zip code
under 100\% of the federal poverty line is significantly right-skewed,
with the majority of zip codes only having below 20\% of their
respective populations below this threshold. For comparison, the
histogram below shows the distribution of percentage below 100\% of the
federal poverty line among the Detroit zip codes in the water testing
data set. Here we see the majority of zip codes have 30-40\% of their
population below 100\% of the federal poverty line.

\begin{Shaded}
\begin{Highlighting}[]
\FunctionTok{hist}\NormalTok{(total}\SpecialCharTok{$}\NormalTok{Pct\_U100)}
\end{Highlighting}
\end{Shaded}

\includegraphics{detroit-schools_files/figure-latex/unnamed-chunk-9-1.pdf}

A boxplot of the percentage of the population by zip code below 100\% of
the federal poverty line shows that the average among schools with
elevated lead in their water testing results is slightly higher, 37\%
vs.~35\%, compared to schools with acceptable lead levels. The variance
among schools with elevated lead levels is also smaller, with fewer
values at low percentage of federal poverty level.

\begin{Shaded}
\begin{Highlighting}[]
\FunctionTok{suppressPackageStartupMessages}\NormalTok{(}\FunctionTok{library}\NormalTok{(ggplot2))}
\FunctionTok{suppressPackageStartupMessages}\NormalTok{(}\FunctionTok{library}\NormalTok{(knitr))}

\CommentTok{\#EDA {-} Water Testing Results by Federal Poverty Line}
\FunctionTok{ggplot}\NormalTok{(}\AttributeTok{data =}\NormalTok{ total, }\FunctionTok{aes}\NormalTok{(}\AttributeTok{x=}\FunctionTok{as.factor}\NormalTok{(status), }\AttributeTok{y=}\NormalTok{Pct\_U100,}
                        \AttributeTok{color=}\FunctionTok{as.factor}\NormalTok{(status))) }\SpecialCharTok{+}
  \FunctionTok{geom\_boxplot}\NormalTok{() }\SpecialCharTok{+}
  \FunctionTok{xlab}\NormalTok{(}\StringTok{\textquotesingle{}Water Testing Result\textquotesingle{}}\NormalTok{) }\SpecialCharTok{+}
  \FunctionTok{ylab}\NormalTok{(}\StringTok{\textquotesingle{}Pct Under 100\% Federal Poverity Line\textquotesingle{}}\NormalTok{) }\SpecialCharTok{+}
  \FunctionTok{ggtitle}\NormalTok{(}\StringTok{\textquotesingle{}Water Testing Result vs \%\textless{}100\% Federal Poverty Line\textquotesingle{}}\NormalTok{)  }\SpecialCharTok{+} 
  \FunctionTok{theme}\NormalTok{(}\AttributeTok{plot.title =} \FunctionTok{element\_text}\NormalTok{(}\AttributeTok{hjust =} \FloatTok{0.5}\NormalTok{), }\AttributeTok{legend.position =} \StringTok{"none"}\NormalTok{)}
\end{Highlighting}
\end{Shaded}

\includegraphics{detroit-schools_files/figure-latex/unnamed-chunk-10-1.pdf}

We'll finish exploratory data analysis by cleaning and examining the
years the school has been opened and number of grades at the school.

\begin{Shaded}
\begin{Highlighting}[]
\FunctionTok{suppressPackageStartupMessages}\NormalTok{(}\FunctionTok{library}\NormalTok{(stringr))}
\FunctionTok{suppressPackageStartupMessages}\NormalTok{(}\FunctionTok{library}\NormalTok{(lubridate))}
\FunctionTok{suppressPackageStartupMessages}\NormalTok{(}\FunctionTok{library}\NormalTok{(knitr))}

\CommentTok{\#number of grades at the schools}
\NormalTok{total}\SpecialCharTok{$}\NormalTok{num\_grades }\OtherTok{\textless{}{-}} \FunctionTok{str\_count}\NormalTok{(total}\SpecialCharTok{$}\NormalTok{EntityActualGrades, }\StringTok{\textquotesingle{},\textquotesingle{}}\NormalTok{) }\SpecialCharTok{+} \DecValTok{1}

\CommentTok{\#number of years the school has been open}
\NormalTok{total}\SpecialCharTok{$}\NormalTok{years\_open }\OtherTok{\textless{}{-}} \FunctionTok{year}\NormalTok{(}\FunctionTok{ymd}\NormalTok{(}\FunctionTok{Sys.Date}\NormalTok{()))  }\SpecialCharTok{{-}} \FunctionTok{year}\NormalTok{(}\FunctionTok{ymd}\NormalTok{(}\FunctionTok{as.Date}\NormalTok{(total}\SpecialCharTok{$}\NormalTok{EntityOpenDate)))}

\CommentTok{\#table(total$EntityAuthorizedEducationalSett, useNA = "always")}

\CommentTok{\#EDA {-} Water Testing Results by Number of Grades}
\FunctionTok{ggplot}\NormalTok{(}\AttributeTok{data =}\NormalTok{ total, }\FunctionTok{aes}\NormalTok{(}\AttributeTok{x=}\FunctionTok{as.factor}\NormalTok{(status), }\AttributeTok{y=}\NormalTok{num\_grades,}
                        \AttributeTok{color=}\FunctionTok{as.factor}\NormalTok{(status))) }\SpecialCharTok{+}
  \FunctionTok{geom\_boxplot}\NormalTok{() }\SpecialCharTok{+}
  \FunctionTok{xlab}\NormalTok{(}\StringTok{\textquotesingle{}Water Testing Result\textquotesingle{}}\NormalTok{) }\SpecialCharTok{+}
  \FunctionTok{ylab}\NormalTok{(}\StringTok{\textquotesingle{}Number of Grades\textquotesingle{}}\NormalTok{) }\SpecialCharTok{+}
  \FunctionTok{ggtitle}\NormalTok{(}\StringTok{\textquotesingle{}Water Testing Result vs Number of Grades\textquotesingle{}}\NormalTok{)  }\SpecialCharTok{+} 
  \FunctionTok{theme}\NormalTok{(}\AttributeTok{plot.title =} \FunctionTok{element\_text}\NormalTok{(}\AttributeTok{hjust =} \FloatTok{0.5}\NormalTok{), }\AttributeTok{legend.position =} \StringTok{"none"}\NormalTok{)}
\end{Highlighting}
\end{Shaded}

\begin{verbatim}
## Warning: Removed 15 rows containing non-finite values (stat_boxplot).
\end{verbatim}

\includegraphics{detroit-schools_files/figure-latex/unnamed-chunk-11-1.pdf}

\begin{Shaded}
\begin{Highlighting}[]
\CommentTok{\#EDA {-} Water Testing Results by Years Open}
\FunctionTok{ggplot}\NormalTok{(}\AttributeTok{data =}\NormalTok{ total, }\FunctionTok{aes}\NormalTok{(}\AttributeTok{x=}\FunctionTok{as.factor}\NormalTok{(status), }\AttributeTok{y=}\NormalTok{years\_open,}
                        \AttributeTok{color=}\FunctionTok{as.factor}\NormalTok{(status))) }\SpecialCharTok{+}
  \FunctionTok{geom\_boxplot}\NormalTok{() }\SpecialCharTok{+}
  \FunctionTok{xlab}\NormalTok{(}\StringTok{\textquotesingle{}Water Testing Result\textquotesingle{}}\NormalTok{) }\SpecialCharTok{+}
  \FunctionTok{ylab}\NormalTok{(}\StringTok{\textquotesingle{}Years Open\textquotesingle{}}\NormalTok{) }\SpecialCharTok{+}
  \FunctionTok{ggtitle}\NormalTok{(}\StringTok{\textquotesingle{}Water Testing Result vs Number of Grades\textquotesingle{}}\NormalTok{)  }\SpecialCharTok{+} 
  \FunctionTok{theme}\NormalTok{(}\AttributeTok{plot.title =} \FunctionTok{element\_text}\NormalTok{(}\AttributeTok{hjust =} \FloatTok{0.5}\NormalTok{), }\AttributeTok{legend.position =} \StringTok{"none"}\NormalTok{)}
\end{Highlighting}
\end{Shaded}

\includegraphics{detroit-schools_files/figure-latex/unnamed-chunk-11-2.pdf}

\begin{Shaded}
\begin{Highlighting}[]
\CommentTok{\#EDA {-} Water Testing Results by Years Open}
\NormalTok{counts }\OtherTok{\textless{}{-}} \FunctionTok{table}\NormalTok{(total}\SpecialCharTok{$}\NormalTok{status, total}\SpecialCharTok{$}\NormalTok{type)}
\FunctionTok{barplot}\NormalTok{(counts, }\AttributeTok{main=}\StringTok{"School Type vs Water Testing Results"}\NormalTok{,}
  \AttributeTok{xlab=}\StringTok{"School Type"}\NormalTok{, }\AttributeTok{col=}\FunctionTok{c}\NormalTok{(}\StringTok{"darkblue"}\NormalTok{,}\StringTok{"red"}\NormalTok{, }\StringTok{"green"}\NormalTok{),}
  \AttributeTok{legend =} \FunctionTok{rownames}\NormalTok{(counts), }\AttributeTok{beside=}\ConstantTok{TRUE}\NormalTok{)}
\end{Highlighting}
\end{Shaded}

\includegraphics{detroit-schools_files/figure-latex/unnamed-chunk-11-3.pdf}

\begin{Shaded}
\begin{Highlighting}[]
\CommentTok{\#cleaning outcome variable}
\NormalTok{total}\SpecialCharTok{$}\NormalTok{elevated }\OtherTok{\textless{}{-}} \FunctionTok{ifelse}\NormalTok{(total}\SpecialCharTok{$}\NormalTok{status}\SpecialCharTok{==}\StringTok{"Elevated"}\NormalTok{, }\DecValTok{1}\NormalTok{, }\DecValTok{0}\NormalTok{)}

\CommentTok{\#drop unneeded variables}
\NormalTok{drop\_columns }\OtherTok{\textless{}{-}} \FunctionTok{c}\NormalTok{(}\StringTok{\textquotesingle{}EntityFIPSCode\textquotesingle{}}\NormalTok{,}\StringTok{\textquotesingle{}EntityActualGrades\textquotesingle{}}\NormalTok{,}\StringTok{\textquotesingle{}EntityPhysicalZip4\textquotesingle{}}\NormalTok{,}\StringTok{\textquotesingle{}school\_name\textquotesingle{}}\NormalTok{,}\StringTok{\textquotesingle{}EntityOpenDate\textquotesingle{}}\NormalTok{,}\StringTok{\textquotesingle{}status\textquotesingle{}}\NormalTok{, }\StringTok{\textquotesingle{}zip\_code\textquotesingle{}}\NormalTok{,}
                  \StringTok{\textquotesingle{}EntityGeographicLEADistrictOff\textquotesingle{}}\NormalTok{, }\StringTok{\textquotesingle{}EntityGeographicLEADistrictOffi\textquotesingle{}}\NormalTok{,}\StringTok{\textquotesingle{}EntityAuthorizedEducationalSett\textquotesingle{}}\NormalTok{)  }
\NormalTok{total }\OtherTok{\textless{}{-}}\NormalTok{ total[ , }\SpecialCharTok{!}\NormalTok{(}\FunctionTok{names}\NormalTok{(total) }\SpecialCharTok{\%in\%}\NormalTok{ drop\_columns)]}
\end{Highlighting}
\end{Shaded}

Finally, we'll tackle class imbalances. We see below that there are many
fewer schools with `Acceptable' vs.~`Elevated' water testing results,
77\% vs.~22\%, respectively (1\% of schools chose not to participate).
Give we're interested in classification, I'll use over sampling the
minority class.

\begin{Shaded}
\begin{Highlighting}[]
\FunctionTok{suppressPackageStartupMessages}\NormalTok{(}\FunctionTok{library}\NormalTok{(caret))}

\NormalTok{tabl1 }\OtherTok{\textless{}{-}} \FunctionTok{table}\NormalTok{(total}\SpecialCharTok{$}\NormalTok{elevated)}
\FunctionTok{kable}\NormalTok{(tabl1, }\AttributeTok{caption =} \StringTok{"Water Testing Results"}\NormalTok{)}
\end{Highlighting}
\end{Shaded}

\begin{longtable}[]{@{}lr@{}}
\caption{Water Testing Results}\tabularnewline
\toprule
Var1 & Freq\tabularnewline
\midrule
\endfirsthead
\toprule
Var1 & Freq\tabularnewline
\midrule
\endhead
0 & 141\tabularnewline
1 & 42\tabularnewline
\bottomrule
\end{longtable}

\begin{Shaded}
\begin{Highlighting}[]
\CommentTok{\#oversampling majority class}
\FunctionTok{set.seed}\NormalTok{(}\DecValTok{1123}\NormalTok{)}
\NormalTok{up\_sample }\OtherTok{\textless{}{-}} \FunctionTok{upSample}\NormalTok{(}\AttributeTok{x =}\NormalTok{ total[, }\SpecialCharTok{{-}}\FunctionTok{ncol}\NormalTok{(total)],}\AttributeTok{y =} \FunctionTok{as.factor}\NormalTok{(total}\SpecialCharTok{$}\NormalTok{elevated))                         }
\FunctionTok{table}\NormalTok{(up\_sample}\SpecialCharTok{$}\NormalTok{elevated) }
\end{Highlighting}
\end{Shaded}

\begin{verbatim}
## < table of extent 0 >
\end{verbatim}

\begin{Shaded}
\begin{Highlighting}[]
\CommentTok{\#creating a training and testing set}
\NormalTok{trainIndex }\OtherTok{\textless{}{-}} \FunctionTok{createDataPartition}\NormalTok{(up\_sample}\SpecialCharTok{$}\NormalTok{Class, }\AttributeTok{p =}\NormalTok{ .}\DecValTok{7}\NormalTok{, }\AttributeTok{list =} \ConstantTok{FALSE}\NormalTok{, }\AttributeTok{times =} \DecValTok{1}\NormalTok{)}

\NormalTok{totalTrain }\OtherTok{\textless{}{-}}\NormalTok{ up\_sample[ trainIndex,]}
\NormalTok{totalTest  }\OtherTok{\textless{}{-}}\NormalTok{ up\_sample[}\SpecialCharTok{{-}}\NormalTok{trainIndex,]}

\CommentTok{\#checking class balance}
\FunctionTok{table}\NormalTok{(totalTrain}\SpecialCharTok{$}\NormalTok{Class)}
\end{Highlighting}
\end{Shaded}

\begin{verbatim}
## 
##  0  1 
## 99 99
\end{verbatim}

\begin{Shaded}
\begin{Highlighting}[]
\FunctionTok{table}\NormalTok{(totalTest}\SpecialCharTok{$}\NormalTok{Class)}
\end{Highlighting}
\end{Shaded}

\begin{verbatim}
## 
##  0  1 
## 42 42
\end{verbatim}

\# Methodology

We'll attempt two models: 1) Classification Tree using the rpart
package, and 2) Random Forest model using the random forest package.

\# Results

The classification tree shows that the most important variables are
school type (charter or public) and percent poverty level. Detroit
Public Schools are more likely than charter schools to have elevated
lead schools, and higher percentage of the population under 100\% of the
federal poverty line are more likely to have elevated lead levels.

\begin{Shaded}
\begin{Highlighting}[]
\FunctionTok{library}\NormalTok{(rpart)}
\FunctionTok{library}\NormalTok{(rpart.plot)}

\NormalTok{mod1 }\OtherTok{\textless{}{-}} \FunctionTok{rpart}\NormalTok{(Class }\SpecialCharTok{\textasciitilde{}}\NormalTok{ ., totalTrain)}
\FunctionTok{summary}\NormalTok{(mod1)}
\end{Highlighting}
\end{Shaded}

\begin{verbatim}
## Call:
## rpart(formula = Class ~ ., data = totalTrain)
##   n= 198 
## 
##           CP nsplit rel error    xerror       xstd
## 1 0.29292929      0 1.0000000 1.1919192 0.06974582
## 2 0.09090909      1 0.7070707 0.7070707 0.06794949
## 3 0.03535354      2 0.6161616 0.6565657 0.06674438
## 4 0.02020202      4 0.5454545 0.6262626 0.06591701
## 5 0.01010101      7 0.4848485 0.6262626 0.06591701
## 6 0.01000000      9 0.4646465 0.6161616 0.06562320
## 
## Variable importance
##   Pct_U100 years_open num_grades       type 
##         28         26         24         21 
## 
## Node number 1: 198 observations,    complexity param=0.2929293
##   predicted class=0  expected loss=0.5  P(node) =1
##     class counts:    99    99
##    probabilities: 0.500 0.500 
##   left son=2 (79 obs) right son=3 (119 obs)
##   Primary splits:
##       type       splits as  LR, improve=8.856398, (0 missing)
##       Pct_U100   < 0.2588713 to the left,  improve=4.646552, (0 missing)
##       num_grades < 4.5       to the left,  improve=4.116807, (21 missing)
##       years_open < 73.5      to the left,  improve=2.014535, (0 missing)
##   Surrogate splits:
##       years_open < 73.5      to the left,  agree=0.793, adj=0.481, (0 split)
##       Pct_U100   < 0.4047876 to the right, agree=0.626, adj=0.063, (0 split)
## 
## Node number 2: 79 observations,    complexity param=0.03535354
##   predicted class=0  expected loss=0.3164557  P(node) =0.3989899
##     class counts:    54    25
##    probabilities: 0.684 0.316 
##   left son=4 (22 obs) right son=5 (57 obs)
##   Primary splits:
##       num_grades < 5.5       to the left,  improve=3.760315, (8 missing)
##       years_open < 21        to the right, improve=3.173572, (0 missing)
##       Pct_U100   < 0.3699775 to the right, improve=2.170412, (0 missing)
##   Surrogate splits:
##       years_open < 7         to the left,  agree=0.775, adj=0.2, (8 split)
## 
## Node number 3: 119 observations,    complexity param=0.09090909
##   predicted class=1  expected loss=0.3781513  P(node) =0.6010101
##     class counts:    45    74
##    probabilities: 0.378 0.622 
##   left son=6 (19 obs) right son=7 (100 obs)
##   Primary splits:
##       Pct_U100   < 0.27305   to the left,  improve=5.8179660, (0 missing)
##       years_open < 8.5       to the right, improve=2.4528730, (0 missing)
##       num_grades < 4.5       to the left,  improve=0.9435596, (13 missing)
## 
## Node number 4: 22 observations
##   predicted class=0  expected loss=0.09090909  P(node) =0.1111111
##     class counts:    20     2
##    probabilities: 0.909 0.091 
## 
## Node number 5: 57 observations,    complexity param=0.03535354
##   predicted class=0  expected loss=0.4035088  P(node) =0.2878788
##     class counts:    34    23
##    probabilities: 0.596 0.404 
##   left son=10 (44 obs) right son=11 (13 obs)
##   Primary splits:
##       num_grades < 6.5       to the right, improve=4.458919, (6 missing)
##       years_open < 21.5      to the right, improve=4.438596, (0 missing)
##       Pct_U100   < 0.2679662 to the right, improve=1.478122, (0 missing)
##   Surrogate splits:
##       Pct_U100 < 0.2679662 to the right, agree=0.804, adj=0.231, (6 split)
## 
## Node number 6: 19 observations
##   predicted class=0  expected loss=0.2631579  P(node) =0.0959596
##     class counts:    14     5
##    probabilities: 0.737 0.263 
## 
## Node number 7: 100 observations,    complexity param=0.02020202
##   predicted class=1  expected loss=0.31  P(node) =0.5050505
##     class counts:    31    69
##    probabilities: 0.310 0.690 
##   left son=14 (92 obs) right son=15 (8 obs)
##   Primary splits:
##       years_open < 8.5       to the right, improve=1.671304, (0 missing)
##       num_grades < 5.5       to the left,  improve=1.250643, (13 missing)
##       Pct_U100   < 0.4020645 to the right, improve=0.405000, (0 missing)
## 
## Node number 10: 44 observations,    complexity param=0.01010101
##   predicted class=0  expected loss=0.2954545  P(node) =0.2222222
##     class counts:    31    13
##    probabilities: 0.705 0.295 
##   left son=20 (12 obs) right son=21 (32 obs)
##   Primary splits:
##       Pct_U100   < 0.3227785 to the left,  improve=2.880682, (0 missing)
##       years_open < 19        to the right, improve=2.560606, (0 missing)
##   Surrogate splits:
##       years_open < 73.5      to the right, agree=0.795, adj=0.25, (0 split)
## 
## Node number 11: 13 observations
##   predicted class=1  expected loss=0.2307692  P(node) =0.06565657
##     class counts:     3    10
##    probabilities: 0.231 0.769 
## 
## Node number 14: 92 observations,    complexity param=0.02020202
##   predicted class=1  expected loss=0.3369565  P(node) =0.4646465
##     class counts:    31    61
##    probabilities: 0.337 0.663 
##   left son=28 (9 obs) right son=29 (83 obs)
##   Primary splits:
##       num_grades < 9.5       to the right, improve=1.9314290, (8 missing)
##       years_open < 18.5      to the left,  improve=1.2294360, (0 missing)
##       Pct_U100   < 0.388707  to the right, improve=0.5625418, (0 missing)
## 
## Node number 15: 8 observations
##   predicted class=1  expected loss=0  P(node) =0.04040404
##     class counts:     0     8
##    probabilities: 0.000 1.000 
## 
## Node number 20: 12 observations
##   predicted class=0  expected loss=0  P(node) =0.06060606
##     class counts:    12     0
##    probabilities: 1.000 0.000 
## 
## Node number 21: 32 observations,    complexity param=0.01010101
##   predicted class=0  expected loss=0.40625  P(node) =0.1616162
##     class counts:    19    13
##    probabilities: 0.594 0.406 
##   left son=42 (18 obs) right son=43 (14 obs)
##   Primary splits:
##       Pct_U100   < 0.377681  to the right, improve=1.358135, (0 missing)
##       years_open < 17        to the left,  improve=0.337500, (0 missing)
##   Surrogate splits:
##       years_open < 10.5      to the left,  agree=0.781, adj=0.5, (0 split)
## 
## Node number 28: 9 observations
##   predicted class=0  expected loss=0.3333333  P(node) =0.04545455
##     class counts:     6     3
##    probabilities: 0.667 0.333 
## 
## Node number 29: 83 observations,    complexity param=0.02020202
##   predicted class=1  expected loss=0.3012048  P(node) =0.4191919
##     class counts:    25    58
##    probabilities: 0.301 0.699 
##   left son=58 (9 obs) right son=59 (74 obs)
##   Primary splits:
##       years_open < 18.5      to the left,  improve=2.6965160, (0 missing)
##       num_grades < 5.5       to the left,  improve=1.7672730, (8 missing)
##       Pct_U100   < 0.3969142 to the left,  improve=0.3190694, (0 missing)
## 
## Node number 42: 18 observations
##   predicted class=0  expected loss=0.2777778  P(node) =0.09090909
##     class counts:    13     5
##    probabilities: 0.722 0.278 
## 
## Node number 43: 14 observations
##   predicted class=1  expected loss=0.4285714  P(node) =0.07070707
##     class counts:     6     8
##    probabilities: 0.429 0.571 
## 
## Node number 58: 9 observations
##   predicted class=0  expected loss=0.3333333  P(node) =0.04545455
##     class counts:     6     3
##    probabilities: 0.667 0.333 
## 
## Node number 59: 74 observations
##   predicted class=1  expected loss=0.2567568  P(node) =0.3737374
##     class counts:    19    55
##    probabilities: 0.257 0.743
\end{verbatim}

\begin{Shaded}
\begin{Highlighting}[]
\FunctionTok{rpart.plot}\NormalTok{(mod1)}
\end{Highlighting}
\end{Shaded}

\includegraphics{detroit-schools_files/figure-latex/unnamed-chunk-13-1.pdf}

\# Appendix

\begin{itemize}
\tightlist
\item
  `DPS Water Testing Results':
  \url{https://data.detroitmi.gov/datasets/dps-water-testing-results}
\item
  `Charter and EEA Schools Water Testing Results':
  \url{https://data.detroitmi.gov/datasets/charter-and-eea-schools-water-testing-results}
\item
  `Data from State of Michigan's Center for Educational Performance and
  Information':
  \url{https://data.detroitmi.gov/datasets/2018-2019-schools-eem?geometry=-83.624\%2C42.271\%2C-82.581\%2C42.449}
\item
  `Income to Poverty Ratios in Michigan by Zip Code Tabulation Area,
  2013':
  \url{https://portal.datadrivendetroit.org/datasets/a57ee4c6fdd24cd686b2305f2e5bf2a8_0}
\end{itemize}

\end{document}
